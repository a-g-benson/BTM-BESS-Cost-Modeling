\section{Cleaning the TTS Dataset and Linking with SGIP}\label{apdx:TTS_SGIP_linking}

In analyses relying primarily on the TTS dataset, I restrict consideration to those observations with a date of installation on or after January 1\Xst, 2000 and DC-rated generating capacity greater than or equal to 1 kilowatt. Next, I trim from the sample those observations with a cost per kilowatt above the top 1\% and below the bottom 1\% in the distribution of cost per kilowatt for the year in which that system was installed. The reason for this is that I observe a sufficiently large number of systems with unrealistic costs per kilowatt that can only be explained by data entry errors. For example, for systems installed in 2020, the first percentile of cost per kilowatt is \$4.2 per kilowatt, which is nearly three orders of magnitude smaller than the median (\$3.85 per watt). There are over 200,000 observations for the year 2020, so this corresponds at least 2,000 such observations with patently implausible data. While not all years exhibit such issues, I trim the sample in all years for consistency and to minimize researcher degrees of freedom.

Because the TTS dataset includes administrative data originating from the various sources it draws upon, it is possible to link observations across TTS and SGIP. To begin the linking procedure, I restrict consideration to the 68,061 observations in TTS which are reported as paired with a BESS. 91.4\% of these projects were installed in the service territory of California IOUs, which reflects the outsized role of SGIP in promoting BTM BESS adoption in the United States.

Ultimately, I am able to link 24,663 observations in the TTS database with corresponding observations in the SGIP sample (as defined in Subsection \ref{sec:data_SGIP}) while 12,190 observations could not be matched. This is primarily because the temporal coverage 2022 edition of TTS includes up to the end of calendar year 2021 (for most data sources). However, 36\% of unmatched SGIP observations have an interconnection date prior to 2022. This cannot be meaningfully explained by the presence of ``standalone'' (not paired with distributed generation) BESS in the SGIP sample. I identify 4,118 BTM solar + storage systems installed prior to 2022 for which TTS does not provide the necessary ID variable to link observations between the two datasets. Per correspondence with \citeauthor{TTS2022}, the lack of an SGIP ID in these cases are a consequence of either (A) missing or incorrect street address information in the confidential datasets provided to the authors of TTS or (B) the observation in question having too many ID variables from too many sources to report them all (TTS only reports two ID variables). I salute \citeauthor{TTS2022} for their Herculean efforts in the linking they were able to achieve.

In the cases where linking was not possible or the data on AC/DC coupling is missing, I impute whether a BESS is AC-coupled or DC-coupled from the manufacturer of the battery. Most manufacturers specialize in one or the other type of battery. For example, all Tesla Powerwalls include an integrated inverter, therefore if SGIP reports that the manufacturer is Tesla, I impute a value of 1 for the variable $AC_i$. This imputation is applied only to observations that are identified as being coupled with some form of distributed generation, either by the SGIP or TTS dataset.