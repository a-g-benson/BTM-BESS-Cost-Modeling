\section{Cumulative Installer Experience Outside of SGIP Utilities}\label{apdx:LBD}

In Section \ref{sec:LBD}, I compute an installing firm's cumulative experience based solely on the observations available in SGIP. Here, I explain my reasoning for the assumption that neglect of experience gained through installations not linked to SGIP participation contributes to negligible measurement error.

I rule out the likelihood of a substantial number of BTM BESS installations elsewhere in California for two reasons. First, I searched the Database of State Incentives for Renewables \& Efficiency (DSIRE)\footnote{\url{https://www.dsireusa.org/}. Search conducted on March 10\Xth, 2022.} and found that no other California utilities historically offered or currently offer a program comparable to SGIP. Secondly, the regions of California that are eligible for SGIP include those (A) in the service territory of the three major IOUs, which represent ~75\% of the electric demand in California, and (B) customers of electric POUs served by natural gas POUs, who account for essentially all of major load centers not served by electric IOUS. This accounts for essentially all of the state's major population centers. Given that no other utilities are actively subsidizing SGIP in California, I conclude that the lack of observations on BTM BESS installations elsewhere in California (e.g. the far north of the state) represent a negligible source of measurement error in the computation of an installer's cumulative experience.

The scale of BTM BESS deployment in the United States outside of California appears to be quite limited: 91.5\% of BTM BESS systems listed in the TTS database are located in utility service territory covered by SGIP. This is far from a perfect estimate of the dominant role of SGIP given that TTS has an estimated coverage of only 78\% of BTM solar PV systems in the United States. For example, TTS lists no BESS installations and only 5 PV installations in Nevada. Nevada has a booming BTM solar industry and NV Energy (the largest utility in the state of Nevada) began offering incentives for BTM BESS 2018. This suggests that the TTS project team has not secured data access from NV Energy.

I focus on installers operating in the two other states bordering California, on the theory that learning-by-doing in BTM BESS installation is embodied in the tacit knowledge of workers and their direct supervisors, rather something that can diffuse outwards from a national corporate headquarters. Through fuzzy matching and direct inspection of the names of installers in the TTS dataset and SGIP dataset, I identify 935 BTM BESS installations in Arizona (out of a total of 2,633) performed by installers who are also present in the SGIP dataset. These are overwhelmingly installations by large corporations operating nationwide---namely Tesla (640) and Sunrun (221)---rather than small and medium enterprises operating in the vicinity of the CA-AZ border. The scope for cross-border operations by Tesla and Sunrun employees based in Arizona is limited in the light of large distances between the major population centers of California and Arizona. I repeat this procedure for Oregon by comparing the SGIP dataset with a comparable dataset from Oregon (described below in \ref{apdx:data_ODOE}; I find zero BTM BESS installers that operate in both Oregon and California. The TTS data set does not have data on Nevada, but given that California's major population centers are far from Las Vegas and Carson City, the scope for cross-border operations is negligible.

\subsection{The Oregon Solar+Storage Rebate Program}\label{apdx:data_ODOE}

The Oregon Solar + Storage Rebate Program (hereafter, ``OSSRP'') was authorized by the state legislature in 2019 and began operation on January 1\Xst, 2020. The program is administered by the Oregon Department of Energy (ODOE).  While earlier statewide programs subsidized the deployment of BTM solar PV, OSSRP it the first statewide program in Oregon to subsidize BTM BESS. A BTM BESS must be paired with solar to qualify for the storage rebate; solar need not be paired with BTM BESS to qualify for the solar rebate. For residential customers, the generosity of the program varies according to whether the customer qualifies as ``low or moderate income,'' which can be determined through enrollment in certain welfare programs or by income verification. For non-residential customers, eligibility is restricted to ``low income service providers": owners of affordable multifamily housing; public, tribal, or nonprofit entities engaged in providing assistive services to households below 100 percent of the state median income; public buildings that provide emergency shelter or otherwise assist with disaster response.

I placed a public records request with ODOE and received the latest version of the administrative data for OSSRP as of October 15\Xth, 2021. I cleaned the data according to procedures similar to those for the SGIP data. The raw number of observations is 1,049, of which fifteen percent include storage while the remainder are PV-only systems. When limiting the sample to applications for rebates that have been approved, the sample size shrinks to 667, of which 83 include storage. This excludes applications that are pending review, have been denied, were withdrawn by the applicant, or have been returned to the applicant by ODOE due to incompleteness or the need for a correction. If the sample is narrowed only to systems which have actually been installed as of October 15\Xth, 2021, then the sample size falls to 365, of which 32 include storage. In light of the small sample size, I do not analyze this data set in the present work. It has only been used for the purposes mentioned above.